\chapter*{ВСТУП}
\addcontentsline{toc}{chapter}{ВСТУП}
\label{1entrance::doc}\label{1entrance:id1}
Підготовка фахівців з нав’язування кібернетичного впливу та захисту вимагає напрацювання теоретичних знань і практичних навичок з використання інструментальних засобів дослідження потенційно-небезпечних дефектів вихідних текстів програм. На сьогодні існує низка засобів для здійснення аналізу та пошуку дефектів, та зазвичай вони є недостатньо інформативними та вимагають попередньої підготовки у використанні ними, тому існує необхідність у створенні більш інтерактивних, масштабованих та інформативних, що сприятиме процесу навчання. Актуальність роботи полягає у створенні програмно-технічних засобів пошуку та дослідження потенційно-небезпечних дефектів вихідних текстів програм.

{\bf Об’єкт дослідження} – підготовка фахівців з аналізу та використання потенційно-небезпечних дефектів вихідних текстів програм.

{\bf Предмет дослідження} – інструментальні засоби  аналізу та використання потенційно-небезпечних дефектіввихідних текстів програм.

{\bf Мета роботи:} Розробити програмий комплекс для відпрацювання практичних навичок фахівців з аналізу та використання потенційно-небезпечних дефектів вихідних текстів програм.
\begin{description}
\item[{{\bf Завдання:}}] \leavevmode\begin{enumerate}
\item {} 
Проаналізувати напрямки підготовки фахівців з нав’язування кібернетичного впливу;

\item {} 
Побудувати програмно-технічний комплекс дослідження цільових програм на переповнення буфера;

\item {} 
Проаналізувати існуючі метрики програмного коду на предмет можливості використання їх для оцінки потенційно вразливих ділянок вихідних текстів програм;

\item {} 
Провести оцінку цінності підготовки фахівців з нав’язування кібернетичного впливу на основі переповнення буферу.
\end{enumerate}
\end{description}